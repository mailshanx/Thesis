\documentclass[12pt,a4paper]{article}
\usepackage[latin1]{inputenc}
\usepackage{amsmath}
\usepackage{amsfonts}
\usepackage{amssymb}
\begin{document}

\section{Schemes, arms, and packets}
We define a \emph{scheme} to be a vector that contains the values of each
parameter we wish to tune. A bandit \emph{arm} is the same as a scheme, except
that it does not specify an FEC. Each arm is associated with a few
different schemes. In our case, each arm is associated with $4$ different
schemes, corresponding to Convolutional coding, Golay coding, both, and none
(test packets).

If a packet encoded in scheme/tentacle $i$ is successfully received, it yields
an ``instantaneous'' data rate $\mu_i$. We have $2$ types of packets: test and
data packets. Test packets are pre-generated bit patterns, useful for estimating
the BER of the corresponding bandit arm. A modem can recieve a test packet $n$
bits in length, and can compute the number of bits $k$ that were erroneously received. 
Since a test packet carries no user information, the corresponding scheme always yields
a data rate of zero. A data packet that is successfully recieved yields a
non-zero data rate depending on the scheme. If the data packet is not
succeffully recieved, it yields a data rate of zero.


\section{Linktuner overview}
The linktuner is used in the context of a point to point communication link
with BER feedback. The master modem decides to switch to a particular scheme,
and transmit packets encoded in that scheme. The slave modem reports
success/failure if data packets were transmitted, and $n$, $k$ if a test packet
was transmitted. The linktuner is fed these two pieces of information, and is
the basis on which it updates it's knowledge.  

The linktuner begins operation by generating an initial set of bandit arms by
uniformly sampling the search space. After $c$ transmissions, we
generate a new set of bandits and add them to our pool of existing ones. The sampling
process draws samples from a probability density function that centers around
the scheme with the highest data rate estimates. As time progresses, the density function
becomes narrower. 

At any given time, the tuner decides which 


\end{document}